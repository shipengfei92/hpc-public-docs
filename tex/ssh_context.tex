\usemodule[zhfonts]
\zhfonts[rm, 12pt]

%% Todo
% 更松弛的行间距;
% 松弛一些目录格式;
% “摘要”样式?

\setupindenting[yes,2em]
\setupheads[indentnext=yes]

\setuptyping
[before={\startframedtext[width=\makeupwidth,
                             background=screen,
                             backgroundscreen=.8]},
    after={\stopframedtext}]

\setupinterlinespace[line=1.4\bodyfontsize]


% Enable hyperlinks
\setupinteraction[state=start, color=middleblue]

\setuppapersize [letter][letter]
\setuplayout    [width=middle,  backspace=1.5in, cutspace=1.5in,
                 height=middle, topspace=0.75in, bottomspace=0.75in]

\setuppagenumbering[location={footer,center}]

\setupbodyfont[12pt]

\setupwhitespace[medium]

\setuphead[chapter]      [style=\tfd]
\setuphead[section]      [style=\tfc]
\setuphead[subsection]   [style=\tfb]
\setuphead[subsubsection][style=\bf]

\setuphead[chapter, section, subsection][number=yes]

\definedescription
  [description]
  [headstyle=bold, style=normal, location=hanging, width=broad, margin=1cm]

\setupitemize[autointro]    % prevent orphan list intro
\setupitemize[indentnext=no]

\setupthinrules[width=15em] % width of horizontal rules

\setupdelimitedtext
  [blockquote]
  [before={\blank[medium]},
   after={\blank[medium]},
   indentnext=no,
  ]

\setupcombinedlist[content][list={chapter,section,subsection,subsubsection}]


\starttext
\startalignment[center]
  \blank[2*big]
  {\tfd 使用SSH登录高性能计算集群}
  \blank[3*medium]
  {\tfa 上海交通大学高性能计算中心\crlf
                \useURL[url1][http://hpc.sjtu.edu.cn][][\hyphenatedurl{http://hpc.sjtu.edu.cn}]\from[url1]}
  \blank[2*medium]
  {\tfa 2013年8月18日 更新}
  \blank[3*medium]
\stopalignment
\placecontent

这份文档将指导您通过SSH远程登录海交通大学高性能计算集群。
在继续阅读本文档前,您应该对\quotation{Linux/Unix}、\quotation{终端}、\quotation{命令提示符}、\quotation{SSH远程登录}有所了解。您可以阅读\in{参考资料}{}[参考资料]中的材料,初步了解这些概念。

这份简短的文档包含了使用SSH登录集群时的注意事项、第一次登录集群时需要做的准备工作---包括信息收集、客户端下载、SSH登录等、SSH文件传输和免密码登录等内容,并附有问题诊断和反馈的建议。

遵循文档的操作规范和反馈方法,将帮助您顺利完成工作。也欢迎大家对文档内容提出建议。谢谢!

\subject[注意事项]{注意事项}

\startitemize[packed]
\item
  用于连接集群的SSH帐号,仅限申请人以及申请人所在的实验室人员使用,不得转借给他人。
\item
  请保存好SSH用户名和密码,切勿告诉他人,高性能计算中心工作人员不会索取用户的SSH密码。
\item
  恶意的SSH客户端,特别是某些\quotation{汉化版客户端},可能会窃取您的SSH密码\footnote{中文版Putty后门事件的报道,请参考
    \useURL[url2][http://www.cnbeta.com/articles/171116.htm][][\hyphenatedurl{http://www.cnbeta.com/articles/171116.htm}]\from[url2]},请使用\in{下载客户端}{}[下载客户端]一节中推荐的英文版SSH软件。
\item
  用户登录到高性能计算集群后,请勿跳转登录到其他节点。使用结束后请及时关闭SSH会话。
\item
  如果连续多次输入错误的SSH密码,或者使用了\quotation{白名单}以外的IP地址连接集群,可能导致您无法登录。请参考\in{问题诊断和反馈}{}[问题诊断和反馈]中的建议自查,并将诊断信息发送给管理员\useURL[url3][mailto:sjtuhpc-sysadmin@googlegroups.com][][\hyphenatedurl{sjtuhpc-sysadmin@googlegroups.com}]\from[url3]。
\stopitemize

\section[准备工作]{准备工作}

\subsection[信息收集]{信息收集}

使用SSH登录到集群时,需要在客户端中填写服务器IP地址(或者主机名)、SSH端口、SSH用户名和SSH密码。
管理员为您分配好账户后,会发送邮件通知您,请查阅该邮件获取这些信息。
邮件片段如下:

\starttyping
SSH login node: 202.120.58.229
SSH Port: 22
Username: YOUR_USERNAME
Password: YOUR_PASSWORD
Home: /lustre/home/YOUR_HOME
\stoptyping

管理员为这位用户分配的的SSH用户名是\type{YOUR_USERNAME},SSH密码是\type{YOUR_PASSWORD},SSH登录节点的IP地址是\type{202.120.58.229},SSH端口是22,用户家目录是
\type{/lustre/home/YOUR_HOME}。

{\em 注意:为叙述方便,下文将沿用这个邮件片段中的登录信息。在实际操作时,请以您收到的邮件为准。建议妥善备份您的登录信息。}

\subsection[下载客户端]{下载客户端}

\subsubsection[windows用户]{Windows用户}

Windows用户请使用\type{putty},这是一个免费绿色的SSH客户端,下载后可双击直接运行。\type{putty}可从其主页
\useURL[url4][http://www.putty.org/][][\hyphenatedurl{http://www.putty.org/}]\from[url4]
下载。

要通过SSH向集群中传输数据,还需要\type{WinSCP}软件。这也是个免费软件,可从其主页
\useURL[url5][http://winscp.net/][][\hyphenatedurl{http://winscp.net/}]\from[url5]
下载安装。

\subsubsection[linuxunixmac]{Linux/Unix/Mac}

Linux/Unix/Mac等*NIX操作系统自带了SSH客户端的命令行工具,包括\type{ssh}、\type{scp}、\type{sftp}等,不需要额外安装软件就能完成SSH登录和数据传输的任务。

\section[使用ssh登录集群]{使用SSH登录集群}

\subsection[windows用户使用putty进行ssh登录]{Windows用户使用Putty进行SSH登录}

Windows用户启动Putty后,请在界面中填写SSH登录节点的地址(\type{IP address})、SSH端口(\type{port}),然后点击\type{Open}连接,如图1所示。

\placefigure[here,nonumber]{在Putty界面中填写SSH地址和端口}{\externalfigure[figures/putty1.png]}

在弹出的终端窗口中,输入SSH用户名和密码进行登录,如图2所示。{\em 注意:在输入密码的过程中,没有\type{*}字符回显提示,请照常输入密码并按回车键确认。}

\placefigure[here,nonumber]{在Putty终端窗口中输入用户名和密码登录}{\externalfigure[figures/putty2.png]}

\subsection[linuxunixmac用户使用命令行工具进行ssh登录]{Linux/Unix/Mac用户使用命令行工具进行SSH登录}

Linux/Unix/Mac用户可以在终端中使用命令行工具登录。下面的命令指定了登录节点的地址、用户名和SSH端口。回车后可按照提示输入密码。

\starttyping
$ ssh -p 22 YOUR_USERNAME@202.120.58.229
\stoptyping

\section[使用ssh传输文件]{使用SSH传输文件}

\subsection[windows用户使用winscp传输文件]{Windows用户使用WinSCP传输文件}

Windows用户可使用WinSCP在自己的计算机和登录节点间传输文件。如图所示,在WinSCP中填写登录节点地址(\type{Host name})、SSH端口号(\type{Port number})、SSH用户名(\type{User name})、SSH密码(\type{Password}),点击\type{Login}后连接。WinSCP使用方法与图形界面的FTP客户端类似,如图4所示。

\placefigure[here,nonumber]{在WinSCP中设定SSH连接参数}{\externalfigure[figures/winscp.png]}

\placefigure[here,nonumber]{WinSCP操作界面}{\externalfigure[figures/winscp2.png]}

\subsection[linuxunixmac用户使用scp和sftp传输文件]{Linux/Unix/Mac用户使用scp和sftp传输文件}

*NIX用户可以使用命令行工具在本地与集群之间传输数据。下面的命令将\type{data/}目录上传到家目录下的\type{tmp/}文件夹下。

\starttyping
$ scp -P 22 -r data/ YOUR_USERNAME@202.120.58.229:tmp/
\stoptyping

下面的命令将家目录下的\type{data.out}下载到本地当前目录。

\starttyping
$ scp -P 22 YOUR_USERNAME@202.120.58.229:data.out ./
\stoptyping

若要进行更复杂的数据传输操作,可以使用\type{sftp},用法与命令行的FTP客户端类似。

\starttyping
$ sftp -P 22 YOUR_USERNAME@202.120.58.229
Connected to 202.120.58.229.
sftp> ls
\stoptyping

\section[免密码交互登录]{免密码交互登录}

{\em 注意:\quotation{免密码交互登录}仅适用于使用SSH命令行工具的Linux/UNIX/Mac用户。}

\quotation{免密码交互登录}不需要在登录时键入用户名和密码,还可以指定服务器别名简化命令。
免密码登录需要建立{\bf 远端主机}(集群登录节点),对{\bf 本地主机}(您的计算机)的SSH信任关系。
建立信任关系后,双方通过SSH密钥对鉴权,无需交互输入密码。
关于SSH密钥对的更多信息,请阅读{[}\#参考资料{]}中的内容。

首先,需要生成本地主机的SSH密钥对。
您可以根据实际需要选择是否使用口令保护密钥对(建议选\quotation{是},且不要选择SSH密码作为口令)。
若选择使用口令保护密钥对,则每次使用私钥进行鉴权时,都需要输入口令。
Mac系统可以自动记住这个口令,使用很方便;Linux/UNIX用户可使用\useURL[url6][https://wiki.gentoo.org/wiki/Keychain][][keychain]\from[url6]辅助SSH密钥管理。

\starttyping
$ ssh-keygen -t rsa
\stoptyping

\type{ssh-keygen}会在\type{~/.ssh/}下生成一对SSH密钥对,其中\type{id_rsa}是私钥,请妥善保存;\type{id_rsa.pub}是公钥,可作为自己的\quotation{身份}公布出去。

然后,使用\type{ssh-copy-id}将本地主机的公钥\type{id_rsa.pub}加入远端主机的信任列表中。
\type{ssh-copy-id}所做的工作,就是把本地主机\type{id_rsa.pub}的内容,添加到远端主机的\type{~/.ssh/authorized_keys}文件中。
在命令执行的过程中,需要您键入SSH密码。

\starttyping
$  ssh-copy-id -p 22 YOUR_USERNAME@202.120.58.229
\stoptyping

我们还可以将连接参数写入\type{~/.ssh/config}中,使得连接命令更加简洁和隐秘。
编辑或新建\type{~/.ssh/config}文件:

\starttyping
$ EDIT ~/.ssh/config
\stoptyping

加入如下内容。其中,\type{Host}指定远端主机的别名,\type{HostName}为真正的域名或IP地址,\type{Port}指定SSH端口,\type{User}指定SSH用户名。

\starttyping
Host hpc
HostName 202.120.58.229
Port 22
User YOUR_USERNAME
\stoptyping

保证这个文件的权限是正确的:

\starttyping
$ chmod 600 ~/.ssh/config
\stoptyping

最后,我们用服务器的别名测试SSH连接。如果使用了密钥口令保护,您需要输入用于保护SSH私钥的口令({\em 注意:不是SSH登录密码})。如果一切妥当,您应该不需要键入SSH密码就登入HPC集群了。

\starttyping
$ ssh hpc
\stoptyping

\section[更改登录密码]{更改登录密码}

登录集群后,您可以使用\type{yppasswd}更改SSH密码。{\em 注意:请不要使用\type{passwd}更改密码,\type{passwd}更改的密码不能在集群中生效。}

\starttyping
$ yppasswd
\stoptyping

如果需要重设密码,请与\useURL[url7][mailto:sjtuhpc-sysadmin@googlegroups.com][][管理员]\from[url7]联系。

\section[问题诊断和反馈]{问题诊断和反馈}

很多原因可能导致您无法通过SSH登录到高性能计算集群,您可以通过如下手段自查。

\startitemize[n][stopper=.]
\item
  访问
  \useURL[url8][http://vpn.hpc.sjtu.edu.cn][][\hyphenatedurl{http://vpn.hpc.sjtu.edu.cn}]\from[url8]
  ,查看自己的外网IP地址。
\item
  通过\type{ping}检查本地主机与登录节点间的网络是否互通:

\starttyping
$ ping 202.120.58.229
\stoptyping
\item
  通过\type{telnet}检查本地主机能否与登录节点的SSH端口建立TCP连接:

\starttyping
$ telnet 202.120.58.229 22
\stoptyping
\item
  查看详细的SSH连接信息:

\starttyping
$ ssh -v YOUR_USERNAME@202.120.58.229
\stoptyping
\stopitemize

如果以上的自查步骤仍无法帮助您解决登录问题,请和\useURL[url8][sjtuhpc-sysadmin@googlegroups.com][][管理员]\from[url8]联系,并将以上的自查结果以及您的用户名附在邮件中。
{\em 建议您直接将自查程序的运行结果以文本复制到邮件中,\quotation{截图}不便于转发和问题追溯,请尽量不要使用。}

\section[在本地显示远端应用程序界面-ssh-x11-forward]{在本地显示远端应用程序界面
SSH X11 Forward}

远端服务器上的图形界面应用程序,还可以通过SSH隧道\quotation{投射}到本地计算机上,这个过程称为SSH
X11 Forward。

要使用SSH X11 Forward,本地计算机需要启动X Server。
使用图形界面的Linux操作系统默认就启动了X Server,Mac OS
X用户可以安装\useURL[url9][http://xquartz.macosforge.org/][][XQuartz]\from[url9],
Windows用户可以安装\useURL[url10][http://www.netsarang.com/products/xmg_overview.html][][Xmanager]\from[url10]。

在建立SSH连接时,加上\type{-X}参数表示启用X11 Forward:

\starttyping
$ ssh -X YOUR_USERNAME@202.120.58.229
\stoptyping

登录到远端服务器后,可以尝试启动一些图形用户界面程序。
如果配置正确,在本地计算机上会弹出应用程序窗口。 在X11
Forward环境中,操作远端应用程序和操作本地应用程序的感觉是一样的,非常方便。

\starttyping
$ xclock
$ gnome-about
\stoptyping

\subject[参考资料]{参考资料}

\startitemize[packed]
\item
  \quotation{UNIX Tutorial for Beginners}
  \useURL[url11][http://www.ee.surrey.ac.uk/Teaching/Unix/][][\hyphenatedurl{http://www.ee.surrey.ac.uk/Teaching/Unix/}]\from[url11]
\item
  \quotation{鸟哥的Linux私房菜:SSH服务器}
  \useURL[url12][http://vbird.dic.ksu.edu.tw/linux_server/0310telnetssh.php\#ssh_server][][\hyphenatedurl{http://vbird.dic.ksu.edu.tw/linux_server/0310telnetssh.php\#ssh_server}]\from[url12]
\item
  \quotation{Simplify Your Life With an SSH Config File}
  \useURL[url13][http://nerderati.com/2011/03/simplify-your-life-with-an-ssh-config-file/][][\hyphenatedurl{http://nerderati.com/2011/03/simplify-your-life-with-an-ssh-config-file/}]\from[url13]
\item
  \quotation{keychain: Set Up Secure Passwordless SSH Access For Backup
  Scripts}
  \useURL[url14][http://www.cyberciti.biz/faq/ssh-passwordless-login-with-keychain-for-scripts/][][\hyphenatedurl{http://www.cyberciti.biz/faq/ssh-passwordless-login-with-keychain-for-scripts/}]\from[url14]
\item
  \quotation{使用putty密码远程登录OpenSSH}
  \useURL[url15][http://www.linuxfly.org/post/175/][][\hyphenatedurl{http://www.linuxfly.org/post/175/}]\from[url15]
\stopitemize

\stoptext
